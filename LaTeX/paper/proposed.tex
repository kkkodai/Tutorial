\chapter[提案手法]{提案手法(改行時のインデントを揃\\\hspace*{2.48cm}えたければ、hspaceを使おう。)}
\section{〇〇ネットワーク}
\label{sec:4.2}
\subsection{〇〇の概要}
\label{sub:4.2.1}
\textbf{\ref{sec:3.2}}節で述べたように従来のデータセットは,音声とそれに付随するモーションであった.
レスト判定手法によって推定されたフェーズラベルを加えて,データセットを三つ組にした.
ジェスチャフェーズとレストフェーズのフェーズラベルの数を表\ref{tb:example}に示す.

\begin{table}[ht] 
\caption{表を作る} 
\label{tb:example}
\hbox to\hsize{\hfil
\begin{tabular}{c|c}\hline
ex1 & ex2 \\\hline\hline
120,962 & 130,489 \\\hline
\end{tabular}\hfil}
\end{table}

\section{〇〇処理}
\label{sec:4.3}
\subsection{従来手法での課題点}
\label{sub:4.3.1}
従来手法のネットワークから出力された〇〇は〇〇が起こった.
そこで,〇〇した.

\subsection{〇〇処理の概要}
\label{sub:4.3.2}
hogehoge

\subsection{予備実験}
\label{sub:4.3.3}

\begin{align}
hogehoge = \frac{1}{N\times64} \sum_{t=0}^N\sum_{i=0}^{64} (||\bar{x}_{t,i} -x_{t,i}||_2)
\label{eq:hogehoge}
\end{align}

ここで,$N$はhogehoge

式(\ref{eq:hogehoge})のように求められる.

\subsection{予備実験の結果}
\label{sub:4.3.4}
hogehoge