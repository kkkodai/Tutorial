% 余白の小さなsectionとsubsectionを定義
% \newcommand{\mychapter}[1]{\vspace{-1pt}\chapter{序論}}
\chapter{序論}
\section{研究背景}
研究背景を書こう

研究背景を書こう

\begin{itemize}
\item 箇条書き1
\item 箇条書き2
\item 箇条書き3
\end{itemize}

\section{研究目的}
\label{sec:1.2}
本研究の目的を以下に列挙する.(ここの書き方は色々あります、列挙しなくても良いです)

\begin{enumerate}
\renewcommand{\labelenumi}{(\alph{enumi})}
\item 本研究の目的1
\item 本研究の目的2
\item 本研究の目的3
\end{enumerate}

目的1
次に,目的2
最後に目的3

\section{本論文の構成}
本論文では評価実験を通して提案手法の有効性を述べる.以下第2章では,関連研究について述べる.
第3章では,○について述べる.
第4章では,○について述べる.
第5章では,○について述べる.
第6章では,評価実験の方法と結果について述べる.
第7章では,考察について述べる.
第8章では,本研究のまとめについて述べる.